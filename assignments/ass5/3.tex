\section{Question 3}

\begin{question}
    Using the \verb+teengamb+ data from the faraway package in R, conduct a hypothesis test to determine whether \verb+verbal+ and \verb+status+ together “affect” \verb+gamble+, controlling for \verb+income+, at $5\%$ significance:
\end{question}

\subsection{Part a}

\begin{question}
    What are the explanatory variables in the big model?
\end{question}

\begin{answer}
    The explanatory variables in the big model is \verb+verbal+, \verb+status+ and \verb+income+.
\end{answer}

\subsection{Part b}

\begin{question}
    What are the explanatory variables in the small model?
\end{question}

\begin{answer}
    The explanatory variable in the small model is \verb+income+.
\end{answer}

\subsection{Part c}

\begin{question}
    What is the F test statistic?
\end{question}

\begin{answer}
    I compute the F test statistic using the code in R
    \begin{verbatim}
        # Fit both models
        lmO2 <- lm(gamble~verbal + status +income, data = teengamb)
        lmo2 <- lm(gamble~income,data = teengamb)
        # Compute the SSE of the both models
        SSEO2 <- (t(lmO2$residuals)%*%lmO2$residuals)[1]
        SSEo2 <- (t(lmo2$residuals)%*%lmo2$residuals)[1]
        # Define n, p, and q
        n2 <- dim(model.matrix(lmo2))[1]
        p2 <- dim(model.matrix(lmO2))[2]
        q2 <- dim(model.matrix(lmo2))[2]
        # Compute the F statistic
        Fstat2 <- ((SSEo2-SSEO2)/(p2-q2))/(SSEO2/(n2-p2))
        Fstat2
    \end{verbatim}
    , and I got a result that is $2.245865$
\end{answer}

\subsection{Part d}

\begin{question}
    What is the p-value?
\end{question}

\begin{answer}
    I found the p-value using 
    \begin{verbatim}
        # Compute the p value
        pval3 <- 1 - pf(Fstat2,p2-q2,n2-p2)
        pval3
    \end{verbatim}
    , and I got the result of p-value $= 0.1181095$.
\end{answer}

\subsection{Part e}

\begin{question}
    What is the conclusion?
\end{question}

\begin{answer}
    First, I found the $F^{*} = 3.21448$ by 
    \begin{verbatim}
        # Compute the Fstar
        Fstar2 <- qf(0.95,p2-q2,n2-p2)
        Fstar2
    \end{verbatim}
    . Then since the $p$-value $0.1181095 > 0.05$, and the F statistic is $2.245865 < 1.21448 = F^{*}$, we know that controlling for \verb+income+, we are $95\%$ confident that \verb+verbal+ and \verb+status+ does not "affect" \verb+gamble+ together.
\end{answer}