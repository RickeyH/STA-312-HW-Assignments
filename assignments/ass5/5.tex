\section{Question 5}

\begin{question}
    Consider the model \verb+gamble+ $\sim$ \verb+sex+ + \verb+income+.
\end{question}

\subsection{Part a}

\begin{question}
    Construct a $95\%$ confidence level interval for the coefficient of \verb+sex+ using any method.
\end{question}

\begin{answer}
    I use the \verb+confint+ command in R as the following code
    \begin{verbatim}
        # Fit the model
        lm1 <- lm(gamble~sex+income, data = teengamb)
        intsex <- confint(lm1,2,level=0.95)
        intsex
    \end{verbatim}
    to construct a $95\%$ confidence level intercal for the coefficient of \verb+sex+, and I got a result of $(-35.35662,-7.912162)$.
\end{answer}

\subsection{Part b}

\begin{question}
    Construct a $95\%$ confidence level boot-strapped interval for the coefficient of \verb+sex+.
\end{question}

\begin{answer}
    I used the following code to find the $95\%$ confidence level boot-strapped interval for the coefficient of \verb+sex+
    \begin{verbatim}
        # Construct a 95% confidence level boot-strapped interval for the coefficient of sex.
        # Find the residuals and fitted value of the original model
        resids <- lm1$residuals
        fit <- lm1$fitted
        # Construct a vector with 4000 entries
        betasexs <- numeric(4000)
        # Construct a for loop to find the boot-strapped interval
        for (i in 1:4000){
          # Randomize the residuals to create 4000 random response variables with errors
          fitboot <- fit + sample(resids,rep = TRUE)
          # Fit new models with 4000 error terms
          lmboot <- lm(fitboot ~ sex + income, data = teengamb)
          # Store the betas for the variable sex
          betasexs[i] <- lmboot$coeff[2]
        }
        # Find the 0.025 and 0.975 quantile of the betas to construct a boot-strapped confidence interval
        cbind(quantile(betasexs,0.025),quantile(betasexs,0.975))
    \end{verbatim}
    , and I got a result of $(-34.38962,-8.711444)$.
\end{answer}