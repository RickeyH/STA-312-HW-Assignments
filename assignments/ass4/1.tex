\section{Question 1}

\begin{question}
    Using the \verb+chredlin+ dataset from the faraway package, do the following:
\end{question}

\subsection{Part a}

\begin{question}
    Define \verb+zips=as.integer(rownames(chredlin))+.
\end{question}

\begin{answer}
    
\end{answer}

\subsection{Part b}

\begin{question}
    Make two models in R:
    \begin{align}
        \Omega &: zips \sim theft + fire,\\
        \omega &: zips \sim fire.
    \end{align}
    
\end{question}

\begin{answer}
    
\end{answer}

\subsection{Part c}

\begin{question}
    Write the two models’ equations. Make sure to keep consistent notation.
\end{question}

\begin{answer}
    
\end{answer}

\subsection{Part d}

\begin{question}
    Construct $95\%$ confidence level intervals for each of the three parameters in $\Omega$ separately, using matrices. Verify your results using \verb+confint+ in R.
\end{question}

\begin{answer}
    
\end{answer}

\subsection{Part e}

\begin{question}
    Find the $t$-scores for each of the three coefficients in $\Omega$ separately. Find the $p$-values for each of the three coefficients in $\Omega$ separately. Verify your results using \verb+summary+ in R.
\end{question}

\begin{answer}
    
\end{answer}

\subsection{Part f}

\begin{question}
    est whether each parameter is separately significant to the model at $\alpha = 0.05$.
\end{question}

\begin{answer}
    
\end{answer}

\subsection{Part g}

\begin{question}
    Determine whether the effect of adding \verb+theft+ to model $\omega$ to make $\Omega$ is statistically significant. Find the $F$ test statistic and $p$-value. How do these compare to (e)?
\end{question}

\begin{answer}
    
\end{answer}

\subsection{Part h}

\begin{question}
    Test whether the combined effect of \verb+fire+ and \verb+theft+ is statistically significant to $\Omega$.
\end{question}

\begin{answer}
    
\end{answer}

\subsection{Part i}

\begin{question}
    Why might rates of fires and thefts be related to zip code as a number (not a category)?
\end{question}

\begin{answer}
    
\end{answer}

\subsection{Part j}

\begin{question}
    lot $95\%$ confidence intervals for each of the two parameters in $\omega$ to make a rectangle.
\end{question}

\begin{answer}
    
\end{answer}

\subsubsection{Part k}

\begin{question}
    Plot a $95\%$ confidence region for the two parameters in $\omega$ together over the same graph.
\end{question}

\begin{answer}
    
\end{answer}