\section{Question 2}

\begin{question}
    Show that the shortest distance from an arbitrary point to an arbitrary line through the origin in $\mathbb{R}^2$ is the length of a vector orthogonal to any vector in the line. Parametrize the line as $\{k[a,b]^T: k \in \mathbb{R}\}$.
\end{question}

\begin{answer}
    Claim: The shortest distance from an arbitrary point to an arbitrary line through the origin in ${\mathbb{R}}^2$ is the length of a vector orthogonal to the line and passing through the point.
    \begin{proof}
    Let $(x_0,y_0)$ be an arbitrary point on ${\mathbb{R}}^2$. We can find the expression of the line in ${\mathbb{R}}^2$, which is $y = \tfrac{b}{a}x \Rightarrow bx - ay = 0$. Then the distance from this point to this line can be calculated as the following:
    \begin{equation}
        \dfrac{\lvert bx_0 - ay_0 \rvert}{\sqrt{b^2 + {(-a)}^2}} = \dfrac{\lvert bx_0 - ay_0 \rvert}{\sqrt{b^2 + a^2}}
    \end{equation}
    
    \end{proof}
\end{answer}
